\documentclass{template/socthesis}

\usepackage{subcaption} 
\usepackage{amsmath} 
\usepackage{enumitem} 
\usepackage{hyperref} % reference
\usepackage{gensymb} % balíček symbolů
\usepackage{booktabs}

\usepackage[toc,page]{appendix}
\usepackage{color} % balíček pro obarvování textů
\usepackage{xcolor}  % zapne možnost používání barev, mj. pro \definecolor
\definecolor{mygreen}{RGB}{0,150,0} % nastavení barev odkazů 
\usepackage{listings} % balíček pro formátování zdrojových kódů 
\usepackage{multirow}
\usepackage{pifont}
\usepackage{pdfpages}
\usepackage{parskip}
\usepackage[author=,status=draft]{fixme} % vkládání poznámek  
% dva módy (status): draft (poznámky se zobrazují v~PDF) / final (poznámky se nezobrazují v~PDF)

\newcommand{\cmark}{\textcolor{green}{\ding{51}}}%
\newcommand{\xmark}{\textcolor{red}{\ding{55}}}%



\lstset { %
    language=C++,
    backgroundcolor=\color{black!5}, % set backgroundcolor
    basicstyle=\footnotesize,% basic font setting
}

\setlist[itemize]{parsep=1pt}

\addbibresource{text.bib} % soubor s~bibliografií
\nocite{*}


% hinty k~používání balíčků hyperref, url, hyperlink a~hypertarget
% \usepackage{hyperref} % balíček pro hypertextové odkazy
% \url{www.odkaz.cz}
% \href{http://www.odkaz.cz}{Text který bude jako odkaz}
% \hyperlink{label}{proklikávací_text} - odkaz na text 
% \hypertarget{label}{cíl_odkazu} - cíl odkazu 

\begin{document} % konec preambule dokumentu


% #########################################################################################
\chapter{Abstrakt}

Cílem mé práce bylo navrhnout ucelený automatický systém monitorující chod pletacích strojů a~přizpůsobit ho co možná nejlépe potřebám výrobní firmy.

Systém jsem navrhoval na míru pro rodinnou firmu na pletení ponožek a~v sou\-čas\-nos\-ti je schopen v~reálném čase zaznamenávat a~následně odesílat naměřená data ze strojů na server. 
Pro uživatele systém nabízí moderní webové stránky, kde si může naměřená data přehledně zobrazit a~analyzovat.

Systém se skládá ze senzorové části, serveru a podpůrného serveru.
Senzorová část je připojená k~pletacímu stroji a~odesílá naměřená data.
Server zajišťuje veškeré zpracování dat a jejich zobrazení uživateli.
Podpůrný server se stará o~aktualizaci a~o~kontrolu správného chodu senzorů.


% #########################################################################################
\chapter{Synopse}



% #########################################################################################
\section{Úvod}
Práce s názvem Integrace do průmyslu 4.0 se zabývá návrhem monitorovacího systému pro firmu na výrobu ponožek.
Celý systém je nasazen v rodinné firmě Rotex XXXXX, která pracuje ve dvousměnném provozu a produkuje měsíčně přes dvanáct tisíc párů ponožek.

Ve firmě je 25 pletacích strojů na ponožky o které se v každé směně starají tři operátoři.
Samotná práce operátora spočívá v kontrole upletených vzorků, otáčením dopletených ponožek a opravách porouchaných strojů.
Poslední z bodů je pro firmu nejkritičtější, zastavení stroje z důvodu poruchy může nastat po několika hodinách ale i v řádech minut.

Můj systém, který jsem interně pojmenoval Pletačka IoT, se stará o monitoring počtu upletených ponožek, dobu zastavení jednotlivých strojů, porovnávání jednotlivých směn, ale také dává zaměstnavateli přehled o aktuální produkci.



% #########################################################################################
\section{Senzory}














\section{Integrace do průmyslu 4.0}
Pojem Průmysl 4.0 se do České republiky dostal okolo roku 2013 a~od té doby se stále více rozšiřuje v~průmyslových firmách.
Jedna z~klíčových částí je IoT (Internet of Things), neboli internet věcí, který nám zajišťuje vzdálenou kontrolu a~řízení strojů pomocí elektroniky, senzorů a~různých softwarů.
Další vlastností těchto systémů je zaznamenávání a~následné ukládání dat do datových úložišť.
Moderní IoT řídící systémy se snaží proniknout co nejvíce do~hloubky a~zpřesnit tak naměřená data, důležitá pro optimalizaci produkce.   

%SECTION
\subsection{Práce operátorů}
Obě dvě pracovní směny se skládají ze tří operátorů. Jeden má na starosti pět až osm pletacích strojů.
Z~každého pletacího stroje vypadne každé čtyři minuty jedna upletené ponožka, kterou operátor musí zkontrolovat a~otočit naruby pro další zpracování.
Dalším úkolem operátorů je doplňování materiálu a~oprava strojů při poruše.


% #########################################################################################
\section{Senzory}

Senzory k~projektu Pletačka IoT jsou postavené na mikročipu ESP32 a~modulu TTGO~T-Display.
Celý tento systém je navržen tak, že na každém pletacím stroji je jeden senzor.
Každý z~těchto senzorů má svoje jedinečné číslo, pod kterým posílá naměřená data na server.
Senzor je napájen z~5 nebo 24~V~a~má spotřebu 120 mA.

\section{Procesor}



% #########################################################################################

\section{Webový server}
Webový server je nasazený na mikropočítači~Raspberry Pi 4 Modelu B.
Toto zařízení jsem zvolil hlavně kvůli nízké spotřebě elektrické energie a~velké komunitě lidí, kteří tento mikropočítač využívají.

Na zařízení běží operační systém Raspberry Pi OS s~grafickým rozhraním.
Webové stránky běží na HTTP serveru Apache2 a~PHP 8.0.
Jako databázový systém využívám MariaDB.
Server běží lokálně uvnitř firmy v~zabezpečené síti, díky čemuž je systém rychlý a~nezávislý na internetovém připojení.


%SECTION
\section{Funkcionalita}
% Přehled veškerých funkcí mého systému.
Níže je popis všech funkcionalit systému.


\subsection{Podrobné statistiky}
Díky tomuto systému má uživatel kompletní přehled o~každém zastavení stroje a~upletené ponožce.


\subsection{Aktuální přehledy}
Na úvodní stránce se vždy zobrazují aktuální přehledy o~průběhu výroby (viz~obrázek \ref{fig:webUvod}).


\subsection{Responzivní design}
Stránka využívá moderní CSS styly, díky kterým se stránka správně zobrazuje na jakémkoliv zařízení. 


\subsection{Chytrý výpočetní algoritmus}
Veškerá naměřená data jsou analyzována mým výpočetním algoritmem.
Algoritmus přijímá uložená data z~databáze, ze kterých postupným procházením vypočítává pracovní statistiky, které následně ukládá do užších výběrů.


\subsection{Jednoduché grafy}
Většina nasbíraných dat se dá přehledně zobrazit v~grafech. 
Díky nim je~porovnávání a~procházení výběrů velmi jednoduché (viz obrázek \ref{fig:webSenzory}).


\subsection{Kontrola běhu stroje}
Prvotním a~nejdůležitějším požadavkem systému bylo sledování běhu stroje.
K~tomu slouží úvodní stránka aplikace, kde uživatel přehledně vidí ikony všech strojů.
Podle barvy ikony dokáže rozeznat, zda stroj běží, nebo~je v~poruše a~následně si může zobrazit další podrobnosti. 

\subsection{Jednoduchý výběr dat}
Každý senzor nabízí jednoduchý výběr dat.
Uživatel si zvolí požadovaný rozsah dat a~aplikace mu tento výběr přehledně zobrazí v~tabulce.


\subsection{Porovnávání směn}
Systém jsem navrhoval tak, aby pracoval s~dvousměnným provozem a~správně přiřazoval data k~jednotlivým směnám.


\subsection{Bezpečné zálohy}
Systém se každý den automaticky zálohuje a~ukládá data na záložní disk, ze kterého se dají případně rychle obnovit.


\subsection{Jednoduchý generátor zkušebních dat}
Pro otestování systému jsem připravil jednoduchý generátor dat, který dokáže simulovat reálné senzory.


% #########################################################################################

\section{Princip fungování Pletačka IoT}
V~předchozích kapitolách byly popsány části systému Pletačka IoT.
V~této kapitole bude celý systém popsán jako celek.


\subsection{Sběr dat}
První, a~tou nejdůležitější částí, je získávání dat pomocí senzorů.
Jakmile senzor zaznamená jakoukoliv změnu, okamžitě tuto zprávu odesílá na server.
Odesílání probíhá skrze senzorové API, kde se nejdříve senzor ověří a~následně se stav zapíše do databáze k~příslušnému senzoru.
Po zapsání do databáze se vrátí do senzoru zpráva o~provedení zápisu. 


\subsection{}{Vyhodnocování dat}
Dalším krokem je zpracovávání surových dat z~databáze.
K~tomuto účelu běží na serveru výběrové API, které je automaticky spouštěné v~nastavený čas.
Jde o~generování širších výběrů dat, hodinové, denní, měsíční a~roční výběry.
Tyto výběry se následně ukládají do databáze k~danému senzoru.
Generování těchto dat probíhá převážně v~noci, kdy je server nejméně vytížen.


\subsection{Zobrazování dat}
Posledním krokem je zobrazení dat uživateli.
Je to jediná část, se kterou se běžný uživatel dostane do kontaktu.
Proto je nutné, aby zobrazení bylo co nejrychlejší a~pro uživatele co nejpříjemnější.
% K~rychlému zobrazování se využívají před generované výběry, ke kterým se rychle dopočítají nově nasbíraná data.
K~rychlému zobrazování se využívají před generované výběry, ke kterým se dopočítají dosud nezpracovaná data a~celý výsledek se zobrazí uživateli.


% #########################################################################################
\section{Závěr}

Cílem této práce bylo navrhnout ucelený systém, který dokáže:

\begin{itemize}
    \item automaticky počítat upletené ponožky
    \item on-line hlásit poruchu na stroji a~zjišťovat celkovou poruchovost strojů
    \item porovnávat výkonnost jednotlivých pracovních směn
    \item monitorovat průběh výroby
    \item nahradí část monotónní práce operátora
    \item zrychlí a~zefektivní výrobu
    \item sníží chybovost
\end{itemize}

Všechny tyto vytyčené cíle se mi podařilo splnit. Systém nadále běží ve~firmě ROTEX Vysočina s.r.o~\cite{ROTEX} a~pomáhá v~běžném provozu.
Můj systém se stal nedílnou součástí výrobního procesu a analyzuje a~zefektivňuje průběh výroby.

Systém je k 1. únoru 2021 nasazen na deseti pletacích strojích a po dobu provozu zaznamenal již přes padesát tisíc upletených ponožek bez závady na~senzorech.

Velkým přínosem pro firmu je porovnávání pracovních směn, díky kterým zaměstnavatel ihned vidí rozdíly mezi produktivitou práce v~daném čase.

Díky SOČ jsem se naučil navrhovat plošné spoje, rozšířil jsem si obzory v~elektronice a~při vývoji jsem si vyzkoušel práci s~měřícími přístroji. 
Také jsem se naučil programovat v~jazyce PHP a~vytvářet komplexní webové systémy.

V~budoucnu bych chtěl tento systém rozšířit na všechny pletací stroje a~pokrýt tak celou výrobu.
Taktéž pokračuji na vylepšování webové aplikace a~plánuji ji rozšířit o~další funkce.
Jde například o~export dat do tabulek.

Tuto práci můžete najít na adrese: \url{https://github.com/JakubAndrysek/SOC-Integrace-do-prumyslu-4.0/blob/master/text.pdf}.

Všechny zdrojové kódy a~DPS k~projektu jsou k~dispozici na \url{https://github.com/Pletacka-IoT} pod MIT licencí.

\newpage


\printbibliography[title=Literatura]

\addcontentsline{toc}{chapter}{Literatura}


\appendix




\end{document}

% Uprava na tvrde mezery "\b([aiouksvz]) " (i~s~tou mezerou na konci) => "$1~"
