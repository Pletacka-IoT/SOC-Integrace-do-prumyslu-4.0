\chapter{Vývoj}
Na této práci jsem začal pracovat v únoru 2020, kdy jsem si jako úplný nováček četl dokumentaci k jazyce PHP. 
Původní verzi webového rozhraní jsem začínal navrhovat v čistém PHP, tento způsob byl však velmi zdlouhavý a neefektivní.
Po měsíci práce v čistém PHP jsem přešel na framework Nette, který mi práci zjednodušil a posunul mě velmi rychle dál. 


%SECTION
\section{Systém Pletačka IoT verze 1.0}
Tato verze vznikla začátkem července kdy už systém uměl pracovat s virtuálními senzory.
Data ze senzorů se zatím pouze ukládala do databáze a web s nimi zatím neuměl pracovat.
Začínal se vyvíjet systém na zpracovávání údajů ze senzorů.


\subsection{Senzory}
Souběžně s programováním webu jsem pracoval na softwaru pro senzory.
V této době byly senzory schopné posílat data na server, ale neměli žádný grafický výstup ani nepodporovaly interakci s uživatelem.

\newpage
%SECTION
\section{Systém Pletačka IoT verze 2.0}
Druhá verze přinesla velké rozšíření systému.
Tato verze přišla v půlce prosince a prošla dlouhodobým testováním.



\subsection{Přizpůsobivost}
Lorem ipsum dolor sit amet, consectetur adipiscing elit.
Aliquam nunc magna, sollicitudin id leo eu, viverra congue risus.
Aliquam consequat ipsum ut erat placerat consequat nec at diam. 
Aenean est odio, molestie sit amet nunc in, pretium luctus elit. 
Donec imperdiet orci vel porttitor placerat. 
Proin ut hendrerit elit, ultricies accumsan urna. 
Vivamus condimentum lorem viverra lectus finibus, nec volutpat turpis auctor.
Cras quis felis non lorem consectetur interdum eu eu sem. 
Proin sit amet feugiat metus. 
Ut vitae orci a enim vestibulum porta. 



%SECTION
\section{Systém Pletačka IoT verze 3.0}
Nadále pracuji na další verzi, která přinese nové funkcionality a vylepší stávající. 

\newpage