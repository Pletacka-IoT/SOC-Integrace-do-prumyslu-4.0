\chapter{Webový server}
Webový server je~nejdůležitější a~nejobsáhlejší část celého systému. 
Webový server mám nasazený na mikropočítači Raspberry Pi 4 Model B který má 8GB operační paměti.
Toto zařízení jsem zvolil hlavně kvůli nízké spotřebě elektrické energie a velké komunitě lidí, kteří tento mikropočítač využívají.
Na zařízení běží operační systém Raspberry Pi OS s grafickým rozhraním.

Webové stránky běží na HTTP serveru Apache2 a PHP 7.3.
Jako databázový systém využívám MariaDB.
Celý webový server mám verzovaný také na Githubu.\newline
Github: \href{https://github.com/Pletacka-IoT/Pletacka-website}{Pletačka-website} 

%SECTION
\section{Fronted}
% Frontend neboli vizuální část webové stránky je pro uživatele ta nejdůležitější.
% Webový design musí zaujmout a udržet si diváka.
% V aktuální době plné mobilních telefonů je důležité aby se aplikace správně zobrazovala i na takto malých zařízeních.
 
Frontend je vizuální část webové stránky zobrazená uživatelem.
Pomocí frontendu se na obrazovku vykreslují veškerý text a jednotlivé prvky stránky.

\subsection{Bootstrap}
Bootstrap je knihovna sloužící k jednoduchému a rychlému vytvoření responzivních webových stránek.
Díky této knihovně jsou stránky správně zobrazovány i na mobilních zařízeních.
Knihovna nabízí propracované grafické a pozicovací příkazy.
Tento nástroj se vyvíjí od roku 2011 a je pod otevřenou licencí.
Webový server využívá Bootstrap ve verzi čtyři.


\subsection{JavaScript}
Na webovém serveru využívám JavaScript primárně k aktualizaci části stránek.
Tato technologie se nazývá AJAX a umožňuje překreslovat vybrané části stránek.
Načítání stánek je tím pádem rychlejší a šetří přenesená data.
K tomuto efektivnímu překreslování používám knihovnu Naja\cite{NAJA}, kterou napsal český vývojář Jiří Pudil.
Knihovna také nabízí jednoduchou integraci do PHP frameworku, o kterém budu psát dále.   



%SECTION
\section{Backend}
Je to nejobsáhlejší část celé této práce. 
Backend je serverová část webové stránky, neběží tedy u vás na počítači jako frontend, ale na webovém serveru.   
Celý backend systému Pletačka IoT jsem napsal v programovacím jazyce PHP a běží na frameworku Nette\cite{NETTE}, který nabízí ucelenou sadu nástrojů k tvorbě webů.
Backend se stará o přijímání dotazů ze senzorů a následný zápis do databáze, pohání celý webový server a pomocí nástroje Cron generuje v nastavených časech databázové výběry.
Nejdříve zde popíšu použité technologie a následně rozeberu princip aplikace.

\subsection{PHP}
Webovou aplikaci programuji v PHP ve verzi 7.3. Jako programovací studio jsem zvolil studentskou verzi aplikace PHPStorm, která je velmi mocným nástrojem při tvorbě webu.
Testovací verze aplikace mám rozjetou na svém počítači kde také celý tento systém vyvíjím. K zasazení aplikace na Raspberry Pi webový server využívám také studio PHPStorm které se pomocí protokolu SSH. 
Funkce také nabízí jednoduché porovnávání změn s produkční verzí a chytrou aktualizaci zdrojových souborů.

Pro snadnější ladění chyb používám Xdebug, díky kterému si můžu krokovat jednotlivé řádky kódu a rychleji tak nalézt chybu.

Jako systém pro správu balíčků používám nástroj Composer, který se ovládá z terminálu pomocí jednoduchých příkazů.
Umožňuje jednoduchou definici závislostí a aktualizaci všech modulů pomocí jednoho příkazu.


\subsection{Nette}
Nette je webový framework vyvíjený komunitou. Vznikl v České republice a jeho zakladatelem je David Grudl.


\subsection{API}
Lorem ipsum dolor sit amet, consectetur adipiscing elit.
Aliquam nunc magna, sollicitudin id leo eu, viverra congue risus.
Aliquam consequat ipsum ut erat placerat consequat nec at diam. 
Aenean est odio, molestie sit amet nunc in, pretium luctus elit. 
Donec imperdiet orci vel porttitor placerat. 
Proin ut hendrerit elit, ultricies accumsan urna. 
Vivamus condimentum lorem viverra lectus finibus, nec volutpat turpis auctor.
Cras quis felis non lorem consectetur interdum eu eu sem. 
Proin sit amet feugiat metus. 
Ut vitae orci a enim vestibulum porta.


%SECTION
\section{Webové rozhraní Pletačka IoT}
Lorem ipsum dolor sit amet, consectetur adipiscing elit.
Aliquam nunc magna, sollicitudin id leo eu, viverra congue risus.
Aliquam consequat ipsum ut erat placerat consequat nec at diam. 
Aenean est odio, molestie sit amet nunc in, pretium luctus elit. 
Donec imperdiet orci vel porttitor placerat. 
Proin ut hendrerit elit, ultricies accumsan urna. 
Vivamus condimentum lorem viverra lectus finibus, nec volutpat turpis auctor.
Cras quis felis non lorem consectetur interdum eu eu sem. 
Proin sit amet feugiat metus. 
Ut vitae orci a enim vestibulum porta.

\subsection{Úvodní stránka}
Lorem ipsum dolor sit amet, consectetur adipiscing elit.
Aliquam nunc magna, sollicitudin id leo eu, viverra congue risus.
Aliquam consequat ipsum ut erat placerat consequat nec at diam. 
Aenean est odio, molestie sit amet nunc in, pretium luctus elit. 


\subsection{Přehled ze senzoru}
Lorem ipsum dolor sit amet, consectetur adipiscing elit.
Aliquam nunc magna, sollicitudin id leo eu, viverra congue risus.
Aliquam consequat ipsum ut erat placerat consequat nec at diam. 
Aenean est odio, molestie sit amet nunc in, pretium luctus elit. 


\subsection{Správa senzorů}
Lorem ipsum dolor sit amet, consectetur adipiscing elit.
Aliquam nunc magna, sollicitudin id leo eu, viverra congue risus.
Aliquam consequat ipsum ut erat placerat consequat nec at diam. 
Aenean est odio, molestie sit amet nunc in, pretium luctus elit. 


\subsection{Nastavení směn}
Lorem ipsum dolor sit amet, consectetur adipiscing elit.
Aliquam nunc magna, sollicitudin id leo eu, viverra congue risus.
Aliquam consequat ipsum ut erat placerat consequat nec at diam. 
Aenean est odio, molestie sit amet nunc in, pretium luctus elit. 



%SECTION
\section{Databáze}
Lorem ipsum dolor sit amet, consectetur adipiscing elit.
Aliquam nunc magna, sollicitudin id leo eu, viverra congue risus.
Aliquam consequat ipsum ut erat placerat consequat nec at diam. 
Aenean est odio, molestie sit amet nunc in, pretium luctus elit. 
Donec imperdiet orci vel porttitor placerat. 
Proin ut hendrerit elit, ultricies accumsan urna. 
Vivamus condimentum lorem viverra lectus finibus, nec volutpat turpis auctor.
Cras quis felis non lorem consectetur interdum eu eu sem. 
Proin sit amet feugiat metus. 
Ut vitae orci a enim vestibulum porta. 


\subsection{Návrh}
Lorem ipsum dolor sit amet, consectetur adipiscing elit.
Aliquam nunc magna, sollicitudin id leo eu, viverra congue risus.
Aliquam consequat ipsum ut erat placerat consequat nec at diam. 
Aenean est odio, molestie sit amet nunc in, pretium luctus elit. 
Donec imperdiet orci vel porttitor placerat. 
Proin ut hendrerit elit, ultricies accumsan urna. 
Vivamus condimentum lorem viverra lectus finibus, nec volutpat turpis auctor.
Cras quis felis non lorem consectetur interdum eu eu sem. 
Proin sit amet feugiat metus. 
Ut vitae orci a enim vestibulum porta. 

\newpage